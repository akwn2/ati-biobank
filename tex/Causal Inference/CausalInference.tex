\documentclass{article}

%% ALEX NAVARRO'S PACKAGES FOR SIMPLE PRE-PRINT ARTICLE (NON-ARXIV)

% Basic typography
\usepackage[margin=2.5cm]{geometry}    % Reduce margins on 1in
\usepackage[bookmarks=false]{hyperref} % Hyperlinks
\usepackage{url}                       % simple URL typesetting
\usepackage{microtype}                 % microtypography
\usepackage{appendix}                  % allow appendices

% Math styling
\usepackage{algorithm}                 % algorithm env.
\usepackage{algorithmic}               % algorithmic env.
\usepackage{amsmath}                   % add AMS mathematical symbols
\usepackage{amsfonts}                  % add AMS fonts
\usepackage{bbm}                       % add blackboard bold fonts
\usepackage[psamsfonts]{amssymb}       % add AMS symbols
\usepackage{nicefrac}                  % compact symbols for 1/2, etc.

% Table and figure styling
\usepackage{booktabs}                  % professional-quality tables
\usepackage{multirow}                  % extended table functionality
\usepackage{dcolumn}                   % extended table functionality
\usepackage{tabularx}                  % extended table functionality
\usepackage{graphicx}                  % modern graphics support

% Drawing packages
\usepackage{tikz}                      % Drawing environment
\usetikzlibrary{shapes,arrows.meta}    % Drawing libraries

% Referencing and revision
\usepackage[noabbrev, capitalise]{cleveref}
\usepackage[backgroundcolor=white,linecolor=red,bordercolor=white,textsize=tiny,textwidth=15mm]{todonotes}


\begin{document}

\title{ATI Biobank Project \\ Causal Inference and tools}

\author{ATI Biobank Group}

\maketitle

\section{Causal Inference methods}
Major methods to look at: Bayesian net, LinGAM
\section{R packages for causal iferenc eand Bayesian network}
The CRAN task view for graphical Models (see \href{https://cran.r-project.org/web/views/gR.html}) lists most popular packages for causal inference and Bayesian network and the relevant ones are listed below:

\begin{enumerate}
\item \url{https://cran.r-project.org/web/packages/gRapHD/index.html}{gRapHD}: Efficient selection of undirected graphical models for high-dimensional datasets
\begin{itemize}
\item Installation requires package graph which is now not available on cran and can be installed by 
\begin{verbatim}
source("https://bioconductor.org/biocLite.R")
biocLite("graph")
\end{verbatim}
\item examples avalaible at \url{https://www.jstatsoft.org/article/view/v037i01}
\item NA not allowed and maximum number of variables can handle 65,000 either discrete or continous.

\end{itemize}
\item \url{https://cran.r-project.org/web/packages/parcor/index.html}{parcor}: estimation of partial correlation.
\item \url{https://cran.r-project.org/web/packages/pcalg/index.html}{pcalg}: Methods for graphical models and causal inference, including the PC algorithm 
\item \url{https://cran.r-project.org/web/packages/abn/index.html}{abn}: Modelling Multivariate Data with Additive Bayesian Networks
\item \url{https://cran.r-project.org/web/packages/bnlearn/index.html}{bnlearn}: Bayesian Network Structure Learning, Parameter Learning and Inference
\item \url{https://cran.r-project.org/web/packages/deal/index.html}{deal}: Learning Bayesian Networks with Mixed Variables

\end{enumerate}
\section{Python}

\section{Others}
\begin{enumerate}
\item \url{http://www.phil.cmu.edu/projects/tetrad/}{TETAD}: Tetrad is limited to models of categorical data (which can also be used for ordinal data) and to linear models ("structural equation models') with a Normal probability distribution, and to a very limited class of time series models. 

\end{enumerate}


\bibliography{references.bib}
\bibliographystyle{ieeetr}
\end{document}
