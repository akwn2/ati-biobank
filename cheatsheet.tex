\documentclass{article}

%% ALEX NAVARRO'S PACKAGES FOR SIMPLE PRE-PRINT ARTICLE (NON-ARXIV)

% Basic typography
\usepackage[margin=2.5cm]{geometry}    % Reduce margins on 1in
\usepackage[bookmarks=false]{hyperref} % Hyperlinks
\usepackage{url}                       % simple URL typesetting
\usepackage{microtype}                 % microtypography
\usepackage{appendix}                  % allow appendices

% Math styling
\usepackage{algorithm}                 % algorithm env.
\usepackage{algorithmic}               % algorithmic env.
\usepackage{amsmath}                   % add AMS mathematical symbols
\usepackage{amsfonts}                  % add AMS fonts
\usepackage{bbm}                       % add blackboard bold fonts
\usepackage[psamsfonts]{amssymb}       % add AMS symbols
\usepackage{nicefrac}                  % compact symbols for 1/2, etc.

% Table and figure styling
\usepackage{booktabs}                  % professional-quality tables
\usepackage{multirow}                  % extended table functionality
\usepackage{dcolumn}                   % extended table functionality
\usepackage{tabularx}                  % extended table functionality
\usepackage{graphicx}                  % modern graphics support

% Drawing packages
\usepackage{tikz}                      % Drawing environment
\usetikzlibrary{shapes,arrows.meta}    % Drawing libraries

% Referencing and revision
\usepackage[noabbrev, capitalise]{cleveref}
\usepackage[backgroundcolor=white,linecolor=red,bordercolor=white,textsize=tiny,textwidth=15mm]{todonotes}


\begin{document}

\title{ATI Biobank Project cheatsheet}

\author{Alex Navarro}

\maketitle

\section{Data\label{sec:data}}
The driving idea behind this section is to have quick access to terminology referring to the dataset as well as some additional pointers for other things related to the main document.

\subsection{Main directories\label{sec:data-dir}}

\begin{itemize}
\item Data file: {\tt /vols/Data/HCP/BBUK/workspace3b.mat}
\item Steve's original readme: {\tt /home/fs0/steve/BB\_README}
\item Data description and lookup: {\tt /vols/Data/HCP/BBUK/SMS/old/ukb6225.html}
\end{itemize}

\subsection{List of main acronyms\label{sec:data-acro}}

General nomenclature
\begin{itemize}
\item ID: subject identity
\item IDP (Imaging-derived phenotypes): Scores derives from brain imaginging ()
\item QC (Quality control): Metrics relating to the ``goodness'' of the data.
\item dMRI (diffusion RMI): A \emph{structural} imaging technique which shows
\item rfMRI (resting-state functional MRI): \emph{functional} or cognitive technique, fMRI imaging with no stimulus
\item tfMRI (task functional MRI) : \emph{functional} or cognitive technique, fMRI while performing some specific task (e.g. recognizing faces)
\end{itemize}

Processing nomenclature
\begin{itemize}
\item SIENAX (Structural Image Evaluation using Normalisation of Atrophy): Technique described in in Smith 2002. It sums up to a measure of the volume of the brain structure adjust the structural image by its atrophy \todo{ALEX: Check this against Stephen's paper, might be reading too much into the name.}
  \item FIRST (FMRIB's Integrated Registration and Segmentation Tool): A tool by Patenaude 2011 for segregating the brain into 15 subcortical structures\todo{ALEX: Cool, but what does it look like in a 3D shape of the Brain? Can we easily localize this so as to visualise it?}
\end{itemize}

\subsection{Data description\label{data}}

Using the UK Biobank Imaging Documentation (UKBID) ---which I found and added to the group folder on Mendeley--- we compared the nomenclature to the data stored within the Matlab variables and Steve's readme description.

The complete list of variables and their related codes can be found on {\tt /vols/Data/HCP/BBUK/SMS/old/ukb6225.html}. This file can be open, e.g. using firefox remotely.

{\tt IDPnames} variable helps to relate the variables to the terminology used in the UKBID. It divides variables in the following groups:
\begin{table}
  \begin{tabular}{lccc}
    \toprule
    {\bf Variable type} & {\bf Matlab Range} & {\bf Type} &{\bf Physical Meaning}\\
    \midrule
    ID  & 1 & None \\
    QC  & 2:17 & Usability score?\\
    IDP T1 SIENAX & 18:28 & Structural & Volumes for brain regions\\
    IDP T1 FIRST & 29:43 & Structural & Volumes for brain regions\\
    IDP SWI $\text{T2}^{*}$ & 44:57 & Structural & Microbleeding count?\\
    IDP tfMRI & 58:73 & Functional & \\
    IDP dMRI TBSS FA & 74:505 & Structural & \\
    IDP dMRI ProbtrackX FA & 506:748 & Structural & \\
    rfMRI amplitudes (ICA 25) node    & 749:769 & Functional &\\
    rfMRI amplitudes (ICA 100) node   & 770:824 & Functional &\\
    rfMRI connectivity (ICA 25) node  & 825:1034 & Functional &\\
    rfMRI connectivity (ICA 100) node & 1035:2519 & Functional &\\
    \bottomrule
  \end{tabular}
  \caption{Type and range for each type of data.}
  \label{tab:data-ranges}
\end{table}
Original numver of subjects: 5847

\section{Statistical and Machine Learning Methods\label{sec:methods}}

The objective of this section is to standardize the terms the group uses to avoid problems when discussing ideas and algorithms. Also, this is meant as a quick guide to the methods we will be using.

\subsection{Nomenclature\label{sec:method-nomen}}
\begin{itemize}
\item Confounding variable - a variable that when conditioned on induces independence on other variables (e.g. $p(x,y)\neq p(x)p(y)$ but $p(x,y|z) = p(x|z)p(y|z)$)
\end{itemize}


\subsection{Models\label{models}}
\begin{itemize}
\item ICA (Independent Component Analysis): for a simple introduction, see \cite{murphy2012} pages 407--416.
\end{itemize}

\bibliography{references.bib}
\bibliographystyle{ieeetr}
\end{document}
